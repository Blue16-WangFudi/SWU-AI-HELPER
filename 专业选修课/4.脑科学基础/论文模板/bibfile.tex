\renewcommand{\baselinestretch}{\yuanbeishu} %
\normalsize\zihao{5}                         %
%\vspace{1cm}                                %
\addcontentsline{toc}{section}{参考文献}     %
\begin{multicols}{2} 
\begin{thebibliography}{9}                   %
%\vspace{.4cm}

\bibitem{1}
Nayar S K, Narasimhan S G. Vision in bad weather[C]//Proceedings of the seventh IEEE international conference on computer vision. IEEE, 1999, 2: 820-827.
\bibitem{2}
	Zhang J, Cao Y, Wang Z. Nighttime haze removal based on a new imaging model[C]//2014 IEEE International Conference on Image Processing (ICIP). IEEE, 2014: 4557-4561.
\bibitem{3}
	Li Y, Tan R T, Brown M S. Nighttime haze removal with glow and multiple light colors[C]//Proceedings of the IEEE international conference on computer vision. 2015: 226-234.
\bibitem{4}
	Pei S C, Lee T Y. Nighttime haze removal using color transfer pre-processing and dark channel prior[C]//2012 19th IEEE International conference on image processing. IEEE, 2012: 957-960.
\bibitem{5}
	Zhang J, Cao Y, Fang S, et al. Fast haze removal for nighttime image using maximum reflectance prior[C]//Proceedings of the IEEE conference on computer vision and pattern recognition. 2017: 7418-7426.
\bibitem{6}
	Liu Y, Wang A, Zhou H, et al. Single nighttime image dehazing based on image decomposition[J]. Signal Processing, 2021, 183: 107986.
\bibitem{7}
	Wang B, Hu L, Wei B, et al. Nighttime image dehazing using color cast removal and dual path multi-scale fusion strategy[J]. Frontiers of Computer Science, 2022, 16(4): 1-13.
\bibitem{8}
	Wang W, Wang A, Liu C. Variational Single Nighttime Image Haze Removal With a Gray Haze-Line Prior[J]. IEEE Transactions on Image Processing, 2022, 31: 1349-1363.
\bibitem{9}
	Zhang J, Cao Y, Zha Z J, et al. Nighttime dehazing with a synthetic benchmark[C]//Proceedings of the 28th ACM international conference on multimedia. 2020: 2355-2363.
\bibitem{10}
	He K, Sun J, Tang X. Single image haze removal using dark channel prior[J]. IEEE transactions on pattern analysis and machine intelligence, 2010, 33(12): 2341-2353.
\bibitem{11}
	Jiang B, Meng H, Ma X, et al. Nighttime image dehazing with modified models of color transfer and guided image filter[J]. Multimedia tools and applications, 2018, 77(3): 3125-3141.
\bibitem{12}
	Kuanar S, Rao K R, Mahapatra D, et al. Night time haze and glow removal using deep dilated convolutional network[J]. arXiv preprint arXiv:1902.00855, 2019.
\bibitem{13}
	Daniel Scharstein and Richard Szeliski. 2003. High-accuracy stereo depth maps using structured light. In 2003 IEEE Computer Society Conference on Computer Vision and Pattern Recognition, 2003. Proceedings., Vol. 1. IEEE, I–I.
\bibitem{14}
	Wang Z, Bovik AC (2006) Modern Image Quality Assessment.Morgan \& Claypool
\bibitem{15}
	Kang S-J (2014) HSI-based color error-aware subpixel rendering technique. J Display Technol 10(11):251–254.
\bibitem{16}
	Sharma G, Wu W, Dalal EN (2004) The CIEDE2000 color-difference formula: implementation notes, supplementary test data, and mathematical observations. Color Res Appl 30(1):21–30.
\bibitem{17}
	Feng, Mengyao, et al. "Learning a convolutional autoencoder for nighttime image dehazing." Information 11.9 (2020): 424.
\bibitem{18}
	Land, E.H. The retinex theory of color vision. Sci. Am. 1977, 237, 108–129.
\bibitem{20}
	Kuanar S, Mahapatra D, Bilas M, et al. Multi-path dilated convolution network for haze and glow removal in nighttime images[J]. The Visual Computer, 2022, 38(3): 1121-1134.
\bibitem{21}
	Cai B, Xu X, Jia K, et al. Dehazenet: An end-to-end system for single image haze removal[J]. IEEE Transactions on Image Processing, 2016, 25(11): 5187-5198.
\bibitem{22}
	Liao Y, Su Z, Liang X, et al. HDP-Net: Haze density prediction network for nighttime dehazing[C]//Pacific Rim Conference on Multimedia. Springer, Cham, 2018: 469-480.
\bibitem{23}
	Zhao D, Li J, Li H, et al. Hybrid local-global transformer for image dehazing[J]. arXiv preprint arXiv:2109.07100, 2021.
%\bibitem{9}
%    许蜜,王跃会,何玉华,李十月,燕虹.
%    \textsl{ARIMA模型在手术部位感染发生率预测中的应用}[J].
%    中华医院感染学杂志,2020,30(01):141-145.
%\bibitem{10}
%    张姝玮,郭忠印,陈立辉.
%    \textsl{基于自回归求积移动平均的制动器温度预测方法}[J].
%    吉林大学学报(工学版),
%    2020,50(06):2080-2086.DOI:10.13229/j.cnki.jdxbgxb20190656.
%\bibitem{11}
%    刘迎,过秀成,周润瑄,吕方.
%    \textsl{基于多源数据融合的干线公交车辆行程时间预测}[J].
%    交通运输系统工程与信息,
%    2019,19(04):124-129+148.DOI:10.16097/j.cnki.1009-6744.2019.04.018.

\end{thebibliography}
\end{multicols}